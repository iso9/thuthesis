
%%% Local Variables:
%%% mode: latex
%%% TeX-master: t
%%% End:
\secretlevel{绝密} \secretyear{2100}

\ctitle{与人相关的图像语义分割}
% 根据自己的情况选,不用这样复杂
\makeatletter
\ifthu@bachelor\relax\else
  \ifthu@doctor
    \cdegree{工学博士}
  \else
    \ifthu@master
      \cdegree{工学硕士}
    \fi
  \fi
\fi
\makeatother


\cdepartment[计算机]{计算机科学与技术系}
\cmajor{计算机科学与技术}
\cauthor{王楠}
\csupervisor{艾海舟教授}
% 如果没有副指导老师或者联合指导老师,把下面两行相应的删除即可。
%\cassosupervisor{陈文光教授}
%\ccosupervisor{某某某教授}
% 日期自动生成,如果你要自己写就改这个cdate
%\cdate{\CJKdigits{\the\year}年\CJKnumber{\the\month}月}

% 博士后部分
% \cfirstdiscipline{计算机科学与技术}
% \cseconddiscipline{系统结构}
% \postdoctordate{2009年7月——2011年7月}

\etitle{Human-Oriented Semantic Image Segmentation}
% 这块比较复杂,需要分情况讨论:
% 1. 学术型硕士
%    \edegree:必须为Master of Arts或Master of Science(注意大小写)
%              “哲学、文学、历史学、法学、教育学、艺术学门类,公共管理学科
%               填写Master of Arts,其它填写Master of Science”
%    \emajor:“获得一级学科授权的学科填写一级学科名称,其它填写二级学科名称”
% 2. 专业型硕士
%    \edegree:“填写专业学位英文名称全称”
%    \emajor:“工程硕士填写工程领域,其它专业学位不填写此项”
% 3. 学术型博士
%    \edegree:Doctor of Philosophy(注意大小写)
%    \emajor:“获得一级学科授权的学科填写一级学科名称,其它填写二级学科名称”
% 4. 专业型博士
%    \edegree:“填写专业学位英文名称全称”
%    \emajor:不填写此项
\edegree{Doctor of Engineering}
\emajor{Computer Science and Technology}
\eauthor{Wang Nan}
\esupervisor{Professor Ai Haizhou}
%\eassosupervisor{Chen Wenguang}
% 这个日期也会自动生成,你要改么?
% \edate{December, 2005}

% 定义中英文摘要和关键字
\begin{cabstract}
图像语义分割是图像分割技术的一种,都是将图像中像素划分为若干区域;而与一般图像分割方法(超像素化)不同的是,它要求产生的区域具有特定语义。因此,图像语义分割是对图像内容的高层解读,并对其进行了精确到像素级的刻画。图像分割具有悠久的研究历史,近年来,语义分割因其对图像内容的精细解析,以及问题本身的复杂性,挑战性,越来越多的研究者进行探索。另一方面,随着人脸与人体检测技术的成熟完善,与人相关的应用已经遍及视频监控,人脸/人体识别,人机交互,游戏娱乐等方方面面,与人们的日常生活联系越来越紧密。但与人相关的分割技术仍不成熟,挑战重重,使得该问题具有高度的学术价值与应用价值。

本文以相对成熟的人脸/人体检测技术为基础,研究了多种与人相关的语义分割问题,包括单物体语义分割、多物体语义分割以及半监督指导下的语义分割,同时也研究了物体形状建模,表观建模以及相关的概率图模型与机器学习方法。本文的主要研究工作包括:

1. 提出了一种对具有复杂形状变化的物体进行形状建模的算法。该算法提出了一种保证输出合理性的子空间聚类约束,以此为基础,建立了融合全局信息的多模态部件模型,并提出了采用势函数有效性优化的部件配置算法。该算法被用于头发区域分割,取得了鲁棒准确的分割结果。

2. 提出了一种对具有遮挡关系的多物体同时进行形状建模的算法。该算法提出了基于相对位置关系与物体自相似特征的遮挡预测方法,并将其引入到多层次的多物体分割框架中,同时提出了分段优化的参数学习算法。该算法应用于多人场景的衣服与行人分割,较好的估计了物体间的遮挡关系,并展示了遮挡关系对于物体分割的作用。

3. 提出了一种结合最近邻分类器与马尔科夫随机场的半监督学习算法。该算法针对分割算法标注数据不足的问题,采用大量未标注数据作为过渡与约束,在超像素基础上,利用无向图来近似高维的物体流形,并允许算法同时对大量图片做分割。该算法用于头发分割问题,证明了半监督学习算法对分割问题的有效帮助。

\end{cabstract}

\ckeywords{物体分割, 形状建模, 表观建模,概率图模型, 半监督学习}

\begin{eabstract}
   Semantic image segmentation is one of image segmentation techniques, which both partition pixels into regions. Its difference from original image segmentation method (e.g. Superpixelization) is that the produced regions are required to be with semantic meanings. Therefore, semantic segmentation is a high-level interpretation of image content and depicts their shapes in pixel level. There is a long research history in image segmentation. Recently, semantic segmentation is explored by more and more researchers due to its subtle representation of image content and intrinsic challenges. On the other hand, as face/human detection techniques develop, human-oriented techniques have been widely applied in many areas, such as vedio surveillance, human-computer interaction, game and entertainments. However, human-oriented segmentation are still immature and challenging, which also brings it high academic and practical values.
   
   In this thesis, based on face/human detection methods, various human-oriented segmentation problems are studied. They are single-object semantic segmentation, multiple-objects semantic segmentation and semi-supervised semantic segmentation. In the same time, object shape modeling, appearance modeling and related graphical models and machine learning methods are also explored. The main work of this thesis includes:
   
   Firstly, a shape modeling algorithm for object with complex shape variations is proposed. This algorithm presents a reasonability-guarantee Subspace-Clustering Dependency as constraints, builds a multi-modal part-based model by incorporating global information and configures the parts by optimizing potential functions' effectiveness. It is applied in hair segmentation problem and achieves robust and accuracy results.
   
   Secondly, a shape modeling algorithm for objects with occlusions is proposed. The algorithm presents an occlusion prediction method based on relative object location and object self-similarity, designs a hierarchical multiple object segmentation algorithm and parameterizes the model by piece-wise optimization. It is applied in clothing and pedestrian segmentation in crowded scenes, making good estimation of the occlusion relationship and showing occlusion reasoning's significance to object segmentation.
   
   Finally, a semi-supervised algorithm combining Nearest Neighbor Classifier and Markov Random Fields is proposed. The algorithm aiming at the insufficient labeled data issue for segmentation problem, in superpixel level, uses large mount of unlabeled data as constraints and undirect graph to approximate high-dimensional object shape manifold and is capable of segmenting multiple images in the same time. It is applied in hair segmentation, showing semi-supervised algorithm's help to segmentation problem.
\end{eabstract}

\ekeywords{object segmentation, shape modeling, appearance modeling, probabilistic graphical model, semi-supervised learning}
